\pagenumbering{gobble}
\newpage
\pagenumbering{Roman}

\addcontentsline{toc}{section}{abstract}
\begin{abstract}

The potential of virotherapy cannot be understated. While there's many novel 
cancer treatments being researched and developed, virotherapy should definetly 
be considered one of the most important. The introduction of a specialised 
virus to tissue affected by cancerous cells shows different effects, all 
leading to cancer treatment without significant side-effets. Three diferent 
viruses were simulated using known data about tumour, cell and virus growth 
and their influence on eachother. All these viruses eventually lead to a 
partial, majority or even complete extermination of cancer cells in tumour 
affected tissue. With that being said, the parameters and virus used are of 
great importance in the effectiveness of the treatment. The viruses vary in 
their infection and destruction rates of the cells. Simulations have also 
shown that the competitiveness of both the cell types are significant in 
determining the effectiveness of treatment. It is important for both the 
cell types to be infected. Absence of competition could eventually lead 
to a premature end to the treatment. This concept of competition is substantiated
by various observations throughout community ecology studies. In this case, the 
normal and tumour cells represent prey populations, who share a common predator,
the virus. Regardless of the potential points of discussion and debate, 
the implication of virus mediated killing of cancer cells
will surely serve as a groundwork for future research.
Virotherapy will potentially lead to a relatively less destructive and more effective treatment.

\end{abstract}
\newpage

\tableofcontents
\listoffigures 
\listoftables

\newpage
\pagenumbering{arabic}